\section{Related Work}

\begin{quote}
\emph{``There’s Plenty of Room at the Bottom.''} – Richard Feynman
\end{quote}

Work on this was inspired after an investigation of Mainland’s paper that implemented a network that took actions based on an energy cost to reward ratio [1]. Primary criticisms included the cost of actual hardware and the relative inefficiency of the network communication. Additionally, we wanted to investigate what OS solutions existed besides TinyOS.

\subsection{Hardware}

\subsection{Network Communication}

\subsection{Sensor Network Operating Systems}

Mainland’s paper simulated the nodes in TinyOS. TinyOS [2]
Nano-RK is an operating system developed at Carnegie-Mellon University. RK stands for Resource Kernel, as it attempts to meet an energy budget by limiting CPU cycles or network usage [3]. 

There are a number of other microcontroller operating systems such as EMERALDS [4], but none could be found that would run on the limited resources of our microcontroller and still have wireless functionality.

ContikiOS [5]