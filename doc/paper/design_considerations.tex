\section{Design Considerations}

Cost remains a major limiting factor in the adoption of sensor networks.  Though the performance-to-price ratio of
hardware is expected to continue to improve, even small differences in price are multiplied across numerous nodes,
and the cookie-cutter manufacturing process of modern digital hardware, which has delivered increasingly powerful
chips for increasingly low costs, also presents cost thresholds to the design process.  This is not likely to change
anytime soon.

We present several protocols designed to create a minimal-effort, minimal-cost, low-power distributed sensor network
which can be implemented with relatively little effort without using external infrastructure, such as micro-kernel or
embedded OSs like EMERALD~\cite{zuberi99} or TinyOS~\cite{tinyos}.  We do this with low-end micro-controllers, such
as the ATtiny~\cite{attinyds}, in mind.

The primary mechanism to save power on a sensor network is to turn the node's radio off.  This presents an inherent
challenge, especially for peer-to-peer systems because when the radio is off, the node is effectively removed from
the network.  We explore two different strategies for tackling this problem, the first, discussed in
Section~\ref{section:heart_beat_protocol} is to dynamically put the nodes
on a synchronized schedule, the other, discussed in Section~\ref{section:random_dc} is to turn the radio on and off
based on duty-cycles which should guarantee an intersection over regular intervals.

