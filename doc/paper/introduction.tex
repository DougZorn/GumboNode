\section{Introduction}

Wireless sensor networks generally consist of a large number of small computers, or nodes, often built using micro-controllers.
There has been increasing interest in being able to construct such a network with and manage it with a fully distributed,
ad-hock protocol.  This has been enabled by advances in the size, power consumption, and overall efficiency of the
core hardware components of the network, along with advances in development and simulation tools.  Sensor networks have
already generated a considerable amount of interest among academic researchers, numerous private companies including
start-ups, and with various government, especially military, organizations.

Monetary cost and power consumption continue to be limiting factors, however, and although the quality of hardware expected to
continue to improve, applications involving wireless sensor networks often require, hundreds, or even thousands of nodes,
even small differences in cost are multiplied.

Furthermore, the cookie-cutter manufacturing process of modern digital hardware, which has allowed the performance-to-price
ratio to improve dramatically over the last few decades, also requires that engineers designing sensor networks pick from
the chips currently available on the market.  This can introduce considerable price-thresholds into the design process.
For example, the difference in cost
between a micro-controller with 8kB and 32kB of flash memory is often only a few dollars, but this could easily represent a
thousands of dollars of cost difference between a network requiring less than 8kB of memory on each node and a network
requiring more than 8kB.  Furthermore, the power consumption of larger chips is greater as well, meaning a few extra
kilo-bytes of memory could also add battery and logistical costs.  Smaller micro-controllers also have less IO pins and
a generally have a slower clock-speed so thresholds exist in around those parameters as well.  Although the overall efficiency of
hardware is expected to improve while cost continues to decrease, wireless sensor engineers will have to contend with price/feature
thresholds for the foreseeable future.

In this paper, we would like to explore the cost of building a low power wireless network.  We particularly wish to
propose a distributed network built using the very lowest-end micro-controllers and wireless chips currently available on the market.
We propose several protocols designed to implement a peer-to-peer, distributed system while minimizing the total cost
of a network, in terms of hardware and power usage.  We then demonstrate the throughput and
other network characteristics possible with these schemes, which are not ideal, but are workable for many applications, and
represents the minimum cost/functionality available with current technology.

We do this with platforms like Arduino~\cite{arduino} in mind.  Arduino is a micro-controller development board built
around open source hardware, designed to be easy to develop on and network together, with one of the stated goals of
the project being to increase the accessibility of multi-disciplinary projects.  Combined with the continued improvement
of micro-kernels, simulators, and other platforms to facilitate the design of sensor networks, we believe the technology
is only becoming more accessible.

The protocols we discuss could be implemented with minimal difficulty, without prior infrastructure, on even the lowest-end
micro-controllers available.  We do this because low-end chips such as the ATtiny~\cite{attinyds} can come with as little as
2kB of memory, which is too small for most micro-kernels, and such chips are rarely given technical support.  We also discuss
implementing them on Contiki~\cite{contiki},
an open-source operating system for small-scale networked devices which is accompanied by an open-source simulator,
Cooja~\cite{cooja}.  With this we hope to demonstrate the capabilities and cost of a minimally-effective distributed
wireless sensor network and how it can be applied to a variety of applications.  We believe this serves as a useful demonstration
of what can be expected from a functional, bare-bones, distributed, wireless sensor network and the cost and
difficulty of implementing it.

%Wireless sensor nodes consist of low-cost micro-controllers with a few sensors attached. Their low cost, small size, and low power consumption make them extremely attractive for deployment over large areas or areas inaccessible to humans. However, this benefit precludes the ability to repair or maintain a specific individual node. As such, nodes need to be as cheap and power efficient as possible.
